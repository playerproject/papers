\documentclass[]{article}

\usepackage{graphicx}
%\usepackage{epsf}
\graphicspath{{./graphics/}}

\usepackage{fullpage}
%\usepackage{vmargin}
%\setmarg{13mm}{4mm}{180mm}{240mm}

\newcommand{\degrees}{$^o$}

\newcommand{\kasper}{{Kasper St{\o}y}}
\newcommand{\richard}{{Richard T.~Vaughan}}
\newcommand{\gaurav}{{Gaurav S.~Sukhatme}}
\newcommand{\maja}{{Maja J.~Matari\'{c}}}
\newcommand{\brian}{{Brian P.~Gerkey}}

\begin{document}

%don't want date printed

\title{ Player and Stage }

\author{\richard \thanks{Corresponding author} 
\and \brian \and \kasper \and \gaurav \and \maja 
\vspace{2mm} \\ 
{\small vaughan, gerkey, kaspers, gaurav, mataric@robotics.usc.edu} 
\vspace{2mm} \\
Robotics Research Laboratories, Department of Computer Science\\
University of Southern California, Los Angeles, CA~~90089-0781
}

%\date{}
\maketitle
%\thispagestyle{empty}

%----------------------------------------------------------------------------
\begin{abstract}

\end{abstract}

%----------------------------------------------------------------------------
\section{Introduction}

isa( player, networked transducer interface )\\

isa( stage, many-multiple transducer simulator that plugs into player)\\
\\
goals for both \\
- enables rapid dev/test of robot controllers \\
- complete and powerful - for researchers\\
- easy to use - beginning reseachers and undergraduate teaching\\
- not prescriptive of controller styyle\\
- sensor-rich\\
- exandable to include more stuff\\
- not tied to any one robot (though was developed on the popular P2)\\
\\
recent developments in equipment and much interest in distributed\\
systems requires very-networked sensor/actuator interface and\\
simulation tools.\\
\\
This paper describes\\
- the need for our software\\
- the software\\
- the development process / history of the software\\
- some detailed examples of its use\\
- (brief) examples of 3rd-party developments\\
- our intended immediate upgrades\\
- where the software will go in future\\

\subsection{Why the names?}
Player was originally named Golem, and the Stage simulator was originally
named Arena.  However, we soon discovered that many,
many, many pieces of robotics-related software already use those names.
So, we had to make a change.  We needed names that capture the now-integral
relationship between the server and simulator, so we chose Player and
Stage, as suggested by a living Englishman, in reference to a
very dead Englishman.

\vspace*{1em}

From {\sl As You Like It} Act II, Scene 7:
\begin{quote}
  ``All the world's a stage, \\
  And all the men and women merely players: \\
  They have their exits and their entrances; \\
  And one man in his time plays many parts,''\\
\end{quote}

From {\sl Macbeth} Act V, Scene 5:
\begin{quote}
  ``Life's but a walking shadow, a poor player \\
  That struts and frets his hour upon the stage \\
  And then is heard no more: it is a tale \\
  Told by an idiot, full of sound and fury, \\
  Signifying nothing.''\\
\end{quote}

%----------------------------------------------------------------------------
\section{Related work}
%We didn't use saphira, ayllu, webots, teambots, etc.  (even though
%they're great) because...
Previous work in the area of robot programming interfaces has 
focused primarily on providing a development environment that suits
a particular control philosophy.  For example, Ayllu \cite{Werger00},
which, like Player, can be used to control the Pioneer robots, provides
tools for creating concurrent behaviors and, further, enforces a 
behavior-based control structure.  Similarly, COLBERT/Saphira 
\cite{Konolige97}, which can also control the Pioneer robots (among others), 
is concerned mainly with the construction of fuzzily-blended behavior-based 
control systems \cite{SaffiottiRuspiniKonolige93}.
While such tools are undeniably useful, we believe that implementing them 
at such a low level imposes undue restrictions on the programmer, who should 
have the full power to build his control system in any way he chooses.
So, in stark contrast to Ayllu and Saphira, we made a clear distinction
in Player between the {\sl programming interface} and the {\sl control
structure}, opting for a maximally general programming interface, with the
belief that users would develop their own tools for building their own
control systems (and they did; see Section~\ref{examples:tools}).  
Further, while Ayllu and Saphira restrict the programmer to a single language
(something akin to C, and something akin to LISP, respectively), the TCP
socket abstraction of Player allows for the use of virtually any programming 
language (see Section~\ref{examples:languages}).  Finally, both Ayllu and
Saphira are commercial software products and, as such, we were not at liberty
to modify them to suit our needs (e.g., fix bugs, add new devices); so, we 
wrote our own, free, robot interface.

Clearly, the system
that is most similar to (and certainly some inspiration for) Player
is the {\scshape Trip} server \cite{Jennings98}; the main difference between
the two is that whereas {\scshape Trip} was designed as a sophisticated
server to support extremely simple clients, we strove 
for minimalism in our server and simplicity in our message protocol, 
at the possible expense of causing the client to do more work.

% TODO: add other stuff, maybe CORBA (GISC-based something something)

As for simulators....

%----------------------------------------------------------------------------
\section{Overview}

block diagram of components

%----------------------------------------------------------------------------
\section{Player}
%what is it?
What is Player? Player is a multi-threaded device server.  It 
executes on a computer that is physically connected to one or more 
sensors and actuators (or, collectively, devices) and exposes a 
unified interface through which clients can connect to and 
control them.  Currently, our primary use for Player is for the control
of our ActivMedia
Pioneer 2-DX mobile robots and the devices that can be connected to them,
such as laser rangefinders and pan-tilt-zoom cameras.  We write control
programs as clients to Player; these clients read sensor data
(e.g., laser range data, compass values) from Player and write actuator 
commands (e.g., wheel velocities, camera pan angles) to it.  As we
explain below, Player is built on a device abstraction sufficiently intuitive
that a beginner can quickly and easily develop his own simple clients, 
yet flexible enough that an advanced researcher can write an arbitrarily
complex and powerful control system.
Although we use Player primarily to control our mobile robots and
their accessory devices, it is important to remember that Player is truly
a general-purpose device server that can be (and is being) used to control
non-robotic sensors and actuators.  Of equal importance is the fact
that Player is {\sl not} a control architecture; it does not prescribe
or even encourage any particular structure for client programs.

\subsection{Design}
%design philsophy  - non-prescriptive - no locking, minimal ACK/NACK
%major design decisions
%device abstraction - justify simplifications, e.g. single update rate
In the design of Player, we made two crucial decisions, which we now
describe:
\begin{itemize}
\item client/server architecture
\item UNIX-like device abstraction
\end{itemize}

\subsubsection{Client/Server Architecture}
In deciding on a client/server architecture, we had only to reflect on our 
collective (painful) experience writing controllers for various
robots.  Most often, what \cite{Jennings98} refers to as the ``direct model'' 
is used: all device management is included in the 
programmer's own control program.  Usually, a library of routines to help
with hardware interaction is provided, but this only relieves the programmer
of the responsibility for personally speaking and understanding a variety of
device-specific message protocols.  Interaction with physical devices 
invariably involves rather specific timing constraints, and the direct model
requires the programmer to adhere to these constraints in his own control
program.  Sophisticated tools, such as Ayllu \cite{Werger00}, do
away with timing concerns by providing a scheduler, but at the expense 
of imposing an event loop that might be fundamentally mismatched for the 
problem at hand, yet difficult to modify or extend.  Another solution to the
timing problem is to multi-thread the control program.  While we cannot deny
the utility of multi-threaded programming for the purpose of naturally 
expressing parallel algorithms (in fact, we often write our own control systems
as a set of threads), we believe that it is unnecessary to force the 
programmer to employ such relatively advanced operating systems techniques
because of the low-level details of the hardware.

Another disadvantage inherent in the direct model is that
the programmer is required to develop his code in
the language in which the library is implemented, prohibiting the 
time-honored engineering tradition of using the right tool (i.e.,
programming language) for the job.  Further, the control program 
must generally execute on the machine that is physically connected to the 
devices the programmer wishes to control, in spite of recent advances in 
wireless networking that can and should allow for location-independent 
device access.  Finally,  in the direct model, control of a single device
is usually limited to the single program that has opened it, even though
the data from many devices, such as laser rangefinders, could clearly 
be provided to multiple clients who might want it.
As embedded and mobile sensors continue to proliferate, the ability
to control them concurrently with many clients from anywhere on the 
network will become increasingly important.

We wanted a programming interface that would overcome all these hurdles
to sensible controller design and allow us to program our robots in 
the languages of our choice, from the machines of our choice, and with
complete disregard for (even ignorance of) the low-level details of how 
each device on the robot works and what is involved in managing it.
Thus, we chose to implement Player as a device server.  Player executes
on the robot and handles all device interaction, hiding the details and
exposing a standard TCP socket interface.  The clients, executing anywhere
that has network connectivity to the robot and written in any language
that can control a TCP socket (even Emacs Lisp can do it), connect to Player
and communicate with it by sending and receiving a small set of 
simple messages.  In practice, users quickly develop client libraries for 
creating and managing this TCP connection in the language of their choice
(such utilties currently exist for C++, Java, Tcl/Tk, and Python).  The
programmer, then, can call these simple routines when necessary and ignore 
the networking details; he can instead concentrate on implementing a good 
solution for the problem at hand.

\subsubsection{UNIX-like Device Abstraction}
We chose a UNIX-like device abstraction with the following properties:
\begin{itemize}
\item sensors and actuators are ``devices'' with read/write semantics
\item access to a device is gained by ``opening'' it in the proper mode
(read,write,all); improper modes are ignored (much like reading /dev/null
or writing /dev/zero)
\item devices are configurable (akin to {\tt ioctl()})
\end{itemize}

\noindent We chose it for the following reasons:
\begin{itemize}
\item standard interface to all devices (``read'' from sensors and ``write''
to actuators, {\tt ioctl()} all of them)
\item compartmentalizes different physical devices; only open the ones
you need
\item through multi-threading, driver development is simplified; new
drivers can be written, debugged, and included without worry over 
interactions
\item split up ``logical'' devices intuitively (e.g., split P2OS into
sonar, position, misc)
\item easy to add feature detectors or other arbitrary pre- and
post-processing as new devices; again, only activated on request
\end{itemize}

%Player is currently specifically implemented
%for the ActivMedia Pioneer 2-DX mobile robot and its most common accessories, 
%but the extensible server architecture should allow simple porting to other 
%devices, and the 

\subsection{Architecture}
\label{architecture}
The Player server has the following internal structure: threads, buffers, etc.

%\begin{figure}[t]
 %\centering
 %\parbox[h]{50mm}{\epsfxsize=0.900\textwidth \epsfbox{buffers.eps}}
 %\caption{{\sl Overall system architecture of Player}}
 %\label{figure:buffers}
%\end{figure}


%architecture


%how does it work?
%what does it do?

%----------------------------------------------------------------------------
\section{Stage}

what does it do?\\
how does it work?\\
design philsophy  - simple to enable fast\\
major design decisions\\
architecture\\
interface to player\\

%----------------------------------------------------------------------------
\section{Performance}

compare running system to stated goals \\
\\
- max effective population vs. sensor use in stage.\\
- bandwith use in player\\
\\
%----------------------------------------------------------------------------
\section{Development history}

- started with Arena and P1s\\
- arenaserver helped us transfer work to P2 robots, Arena updated to match\\
- arenaserver was closer to what we wanted than our existing interfaces\\
- in-house papers published using Arena/Arenaserver\\
- SCOWR work done in Arena, papers published\\
- Golem developed on the robots\\
- Arena interfaced to Golem\\
- names changed\\
- this paper written\\

%----------------------------------------------------------------------------
\section{Case studies}

- Ant\\
- Publish-subscribe\\

%----------------------------------------------------------------------------
\section{Other projects using the software}
\subsection{Client Utilities}
\label{examples:languages}
- C++ \\
- Java \\
- Tcl/Tk \\
- Python \\

\subsection{Tools/Projects}
\label{examples:tools}
- SCOWR verious\\
- RESL everything - Boyoon's gadgets\\
- Gaurav'a class projects\\
- Andrew's viewer and device mods\\

%----------------------------------------------------------------------------
\section{Future work}

immediate\\
- more sensors\\
- more robots (run-time Stage robot configuration)\\
\\
long term\\
- port player to Everything - a platform for ubiquitous Sensor networks?\\
- distribute Stage to allow very large populations\\
\\
%----------------------------------------------------------------------------
\section{Conclusion}

Nobody does it better.

%----------------------------------------------------------------------------
\section*{Obtaining the software}

- get the web site together!

\bibliographystyle{alpha}
\bibliography{robot}

\end{document}

