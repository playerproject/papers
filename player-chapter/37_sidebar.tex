%%%%%%%%%%%%%%%%%%%%%%%%%% author.tex %%%%%%%%%%%%%%%%%%%%%%%%%
%
% sample root file for your contribution to a
%
% "contributed book" (global)
%
% Use this file as a template for your own input.
%
%%%%%%%%%%%%%%%%%%%%%%%% Springer-Verlag %%%%%%%%%%%%%%%%%%%%%%%%%%


%%% The following preamble of the contribution has been commented out
%%% to allow LaTeX to \include that document into the main book

% RECOMMENDED %%%%%%%%%%%%%%%%%%%%%%%%%%%%%%%%%%%%%%%%%%%%%%%%%%%
%\documentclass{svmult}

%% choose options for [] as required from the list
%% in the Reference Guide, Sect. 2.2

%\usepackage{makeidx}         % allows index generation
%\usepackage{graphicx}        % standard LaTeX graphics tool
                              % when including figure files
%\usepackage{multicol}        % used for the two-column index
%\usepackage[bottom]{footmisc}% places footnotes at page bottom
% etc.
% see the list of further useful packages
% in the Reference Guide, Sects. 2.3, 3.1-3.3

%\makeindex             % used for the subject index
                       % please use the style sprmidx.sty with
                       % your makeindex program


%%%%%%%%%%%%%%%%%%%%%%%%%%%%%%%%%%%%%%%%%%%%%%%%%%%%%%%%%%%%%%%%%%%%%

%\begin{document}

\title*{Sidebar - CORBA: Common Object Request Broker Architecture}
% Use \titlerunning{Short Title} for an abbreviated version of
% your contribution title if the original one is too long
\author{Davide Brugali}
% Use \authorrunning{Short Title} for an abbreviated version of
% your contribution title if the original one is too long
\institute{Universit\'a degli Studi di Bergamo, Italy
\texttt{brugali@unibg.it} }
%
% Use the package "url.sty" to avoid
% problems with special characters
% used in your e-mail or web address
%
\maketitle

CORBA (Common Object Request Broker Architecture) is a standard
for distributed middleware defined by the Object Management Group
(OMG), a consortium of more than 700 organisations including
software industry leaders such as Sun, HP, IBM, Microsoft and
Rational. This architecture has reached a good level of maturity
and is now implemented in more than ten commercial products.

The basic component of the architecture is the Object Request
Broker (ORB), which should be installed on each connected host.
The ORB uses the Stub/Skeleton mechanism for remote communication
between client and server objects. The stub and the skeleton of a
server object are generated at compile-time from a declarative
specification of the server's interface in the language-neutral
Interface Definition Language (IDL). The interface describes which
methods the server object supports, which events the object can
trigger, and which exceptions the object raises. The server object
can be implemented in any programming language (C, C++, Java,
etc.). The ORB installed on the server side is in charge of
translating the incoming IDL service requests from the remote
clients into the server's method invocations. The IDL supports the
declaration of only basic type arguments, which have to be passed
to a remote server. The connection between the stub and the
skeleton is established using the Remote Procedure Call (RPC)
mechanism.

The ORB supports location transparency, as it provides the client
with a reference to the server object regardless of its network
location. The client side ORB dispatches service requests to the
server side ORB transparently. Since CORBA can be implemented
using different technologies, the Internet Inter-ORB Protocol
(IIOP) defines the standard communication protocol for
inter-vendor ORB compatibility.

The ORB supports the control of the threading policy used by the
servers, such as one-thread-per request and
one-thread-per-objects. This threading capability is at the basis
of recent CORBA extensions towards a real-time ORB (see
\cite{CORBA-RT} for a comprehensive survey on recent results in
developing a standard real-time CORBA).

Server objects can publish the IDL specifications of their
interface using the Interface Repository API. The Interface
Repository is used to implement two services for object location.
The Trader Service is similar to the yellow pages. It allows
client objects to find out which distributed server objects
support a given interface. The Trader Service returns a list of
references to the objects, which have registered that interface
and a description of the usage constraints for each object. The
Naming Service allows client objects to identify which interface
is supported by an object that has been registered with a given
name. In some cases, the client object has to request the service
of a server object, whose interface was not known at compile-time.
This means that the client object does not have a stub of the
remote server object, but it can use a generic interface, called
Dynamic Invocation Interface that makes it possible to construct
method invocations at run-time.

Other common services are the Persistent object service for
storing and retrieving objects in/from a persistent storage, the
Transaction service, the Life-cycle service to create, delete,
copy and move objects, and the Event service for asynchronous
communication among distributed objects. The real time CORBA
standard specifies an optional Scheduling service, which uses the
primitives of the real-time ORB to achieve a uniform scheduling
policy in CORBA systems.


%%%%%%%%%%%%%%%%%%%%%%%%%%%%%%%%%%%%%%%%%%%%%%%%%%%%%%%%%%%%%%%%%%%%%%

%\printindex
%\end{document}
