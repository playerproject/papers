%%%%%%%%%%%%%%%%%%%%%%%%%% author.tex %%%%%%%%%%%%%%%%%%%%%%%%%
%
% sample root file for your contribution to a
%
% "contributed book" (global)
%
% Use this file as a template for your own input.
%
%%%%%%%%%%%%%%%%%%%%%%%% Springer-Verlag %%%%%%%%%%%%%%%%%%%%%%%%%%


%%% The following preamble of the contribution has been commented out
%%% to allow LaTeX to \include that document into the main book

% RECOMMENDED %%%%%%%%%%%%%%%%%%%%%%%%%%%%%%%%%%%%%%%%%%%%%%%%%%%
%\documentclass{svmult}

%% choose options for [] as required from the list
%% in the Reference Guide, Sect. 2.2

%\usepackage{makeidx}         % allows index generation
%\usepackage{graphicx}        % standard LaTeX graphics tool
                              % when including figure files
%\usepackage{multicol}        % used for the two-column index
%\usepackage[bottom]{footmisc}% places footnotes at page bottom
% etc.
% see the list of further useful packages
% in the Reference Guide, Sects. 2.3, 3.1-3.3

%\makeindex             % used for the subject index
                       % please use the style sprmidx.sty with
                       % your makeindex program


%%%%%%%%%%%%%%%%%%%%%%%%%%%%%%%%%%%%%%%%%%%%%%%%%%%%%%%%%%%%%%%%%%%%%

%\begin{document}

\title*{An Integration Framework for Developing Interactive Robots}
% Use \titlerunning{Short Title} for an abbreviated version of
% your contribution title if the original one is too long
\author{J. Fritsch\inst{1}\and S. Wredea\inst{1}}
% Use \authorrunning{Short Title} for an abbreviated version of
% your contribution title if the original one is too long
\institute{Bielefeld University, Faculty of Technology,, Germany
\texttt{\{fjannik,swredeg\}@techfak.uni-bielefeld.de}}
%
% Use the package "url.sty" to avoid
% problems with special characters
% used in your e-mail or web address
%
\maketitle

\section{Introduction}
\label{sec:ch33-Intro}

Text \cite{33_castek00component}.


\subsection{Subsection}
\label{sec:ch33-subsection}

Text


%%%%%%%%%%%%%%%%%%%%%%%%%%%%%%%%%%%%%%%%%%%%%%%%%%%%%%%%%%%%%%%%%%%%%%

%\printindex
%\end{document}
