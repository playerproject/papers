%%%%%%%%%%%%%%%%%%%%%%%%%% author.tex %%%%%%%%%%%%%%%%%%%%%%%%%
%
% sample root file for your contribution to a
%
% "contributed book" (global)
%
% Use this file as a template for your own input.
%
%%%%%%%%%%%%%%%%%%%%%%%% Springer-Verlag %%%%%%%%%%%%%%%%%%%%%%%%%%


%%% The following preamble of the contribution has been commented out
%%% to allow LaTeX to \include that document into the main book

% RECOMMENDED %%%%%%%%%%%%%%%%%%%%%%%%%%%%%%%%%%%%%%%%%%%%%%%%%%%
%\documentclass{svmult}

%% choose options for [] as required from the list
%% in the Reference Guide, Sect. 2.2

%\usepackage{makeidx}         % allows index generation
%\usepackage{graphicx}        % standard LaTeX graphics tool
                              % when including figure files
%\usepackage{multicol}        % used for the two-column index
%\usepackage[bottom]{footmisc}% places footnotes at page bottom
% etc.
% see the list of further useful packages
% in the Reference Guide, Sects. 2.3, 3.1-3.3

%\makeindex             % used for the subject index
                       % please use the style sprmidx.sty with
                       % your makeindex program


%%%%%%%%%%%%%%%%%%%%%%%%%%%%%%%%%%%%%%%%%%%%%%%%%%%%%%%%%%%%%%%%%%%%%

%\begin{document}

\title*{Research Directions in Robotic Software Frameworks}
% Use \titlerunning{Short Title} for an abbreviated version of
% your contribution title if the original one is too long
\author{Davide Brugali\inst{1}\and Antonio Cisternino\inst{2}
\and Diego Colombo\inst{2} \and Carle C�t�\inst{3}\and Jannik
Fritsch\inst{4}\and Brian Gerkey\inst{5} \and Gerhard
Kraetzschmar\inst{6} \and Fran�ois Michaud\inst{3} \and Richard
Vaughan\inst{7}\and Hans Utz\inst{8}}
\authorrunning{Brugali et al.}
\institute{Universit\'a degli Studi di Bergamo, Italy \texttt{brugali@unibg.it} 
\and AAA \texttt{aaa@bbb.bb} 
\and AAA \texttt{aaa@bbb.bb}
\and AAA \texttt{aaa@bbb.bb}
\and SRI International, USA \texttt{gerkey@ai.sri.com} 
\and AAA \texttt{aaa@bbb.bb}
\and Simon Fraser University, Canada \texttt{vaughan@sfu.ca}
\and AAA \texttt{aaa@bbb.bb}}


%
% Use the package "url.sty" to avoid
% problems with special characters
% used in your e-mail or web address
%
\maketitle

\section{Introduction}
\label{sec:ch31-Intro}

In the software community, a framework indicates an integrated set
of domain-specific software components
\cite{31_CoplienSchmidt1995} which can be reused to create
applications. A framework is more than a library of software
components: It defines the common architecture underlying the
particular applications built on the framework. Frameworks are a
powerful development approach as they consist of both reusable
code (the component library) and reusable design (the
architecture).

Frameworks have acquired popularity in object-oriented (OO). Here,
the interpretation of "framework" ranges from structures of
classes of cooperating objects, which through extension provide
reusable basic designs for a family of similar applications
\cite{31_JohnsonFoote1988}, to the more restrictive view
\cite{31_Schmid1995} of complete, high level modules, which
through customization directly result in specific applications for
a certain domain. Customization is done through parameterization
or by writing functional specifications.

Frequently, the two views of framework, referred to as white box
and black box approaches to reuse, may be simultaneously present
in one framework. In fact, features which are likely to be common
to most applications can be offered and therefore reused as black
boxes. Black-box frameworks support extensibility by defining
interfaces for components that can be plugged into the framework
via composition. Existing functionalities are reused by defining
components that conform to a particular interface and integrating
these components into the framework. On the other hand, the class
library accompanying the framework usually provides base classes
(seen as white boxes) that can be specialized, by adding
subclasses as needed, and easily integrated with the rest of the
framework. White-box frameworks rely heavily on OO language
features like inheritance and dynamic binding in order to achieve
extensibility. Existing functionality is reused and extended by
(1) inheriting from framework base classes and (2) overriding
predefined hook methods using design patterns like the Template
Method \cite{31_Gamma1995}.

Framework based development has two main processes: development of
the framework and development of an application adapting the
framework. Every application framework evolves over time; this is
called the framework life span \cite{31_Brugali1997}. Within this
life span, the basic architectural elements which are independent
from any specific application are implemented first. When new
applications are built using the framework, new components are
developed which eventually are generalized and integrated as new
black box elements. This means that in its early stages a software
framework is mainly conceived as a white box framework. However,
through its adoption in an increasing number of applications, the
framework matures: More concrete components, which provide black
box solutions for common difficult problems, as well as high level
objects, which represent the major abstractions found in the
problem domain, become available within the framework.

Since developing the framework requires a lot of effort, it is
justified only if several applications will be built based on the
framework. On the other hand, when the framework is available,
developing an application from it offers considerable saving in
time and effort compared to the development from scratch. The
application is built by slightly modifying the framework, but by
accepting the framework's high level design in general. In other
words, most of the framework is reused, and reuse happens both at
the high level design and at the code level. Building a framework
is more demanding than building an application. The designer has
to have a deep understanding of the domain the application is
embedded into, and has to anticipate the needs of future
application developers. As a matter of fact, successful frameworks
are often derived from reengineering legacy applications by
abstracting from them the knowledge of principal software
designers.

Most framework-based software development is organized along the
value-added principle \cite{31_Mili2002}. This divides any
software system into a set of horizontal layers, each one built on
top of another. This approach promotes separation of concerns and
enforces modularity. Typically, a software framework is subdivided
into a system infrastructure layer, a functional (also called
horizontal or business domain) layer, and an application (also
called vertical) layer.

The \emph{system infrastructure layer} (usually called
\emph{middleware}) consists of a collection of software components
that offer basic services like communication between distributed
applications, uniform access to heterogeneous resources,
independence from processing platforms and distributed location,
distributed component naming, service brokering and treading, and
remote execution. Examples of distributed middleware systems are
Microsoft .NET, the implementations of OMG CORBA, and the Sun
Enterprise Java Beans.

The customization of the system infrastructure layer is usually
black-box. It consists in the aggregation of the frameworks'
elemental components (such as mechanisms for logical
communication, concurrency, and device abstraction) in which the
whole application has to be written.

The \emph{functional} (also called \emph{business domain})
\emph{layer} is the model of a robotic application and consists of
software components, which directly map to real entities like a
vision system or to common robot functionalities such as sensor
processing, mapping, localization, path planning, and task
planning, or even to typical robot capabilities such as reactive
behaviors.

Typically, they are implemented on top of the system
infrastructure layer. For example, reactive behaviors delegate
device abstractions for motion actuation, sensor processing, path
planning, and task planning rely on the framework concurrency
model in order to execute on different time scale, multi-robot
mapping and localization build on data exchanged through a common
distributed environment.

The customization of the functional layer is usually white-box.
The basic components are intermediary classes, which by their very
nature are fairly application independent, although they have been
conceived bearing in mind a specific application domain (e.g.
robot mobility or visual servoing). They are written using the
elemental components of the framework, and they have to be
specialized for each concrete application.

The \emph{application layer} connects the functional components
according to the information and control flows of any robot task.
Examples of robot tasks are soccer playing, vacuum cleaning,
museum tour guiding, mars surface exploration.

The customization of the application layer is usually grey-box. It
consists of the interconnection of pluggable application-specific
components through the middleware system. These components are the
result of the adoption of the framework for the development of
more and more applications and in some cases they are open source
components-off-the-shelf (COTS). COTS products are designed to
support a standard virtual interface, which consists of a set of
rules for accessing the component's functionality in a homogeneous
way, regardless of the execution platform, the programming
language, and other specifics.

Software frameworks offer a unified view to model and develop
robotic software systems at every level of the vertical
decomposition from the system infrastructure to the final
application through the functional model.

The papers in the third Part of this book describe different
approaches that address to some extent most or all of the aspects
of robot framework development (such as black box or white box
reuse. Individually, they propose innovative contributions on how
to apply framework development concepts in the robotics domain.
These approaches are illustrated in the next section. Thereafter,
the final section draws some conclusions.


\section{Opportunities to exploit Framework-based development in Robotics}
\label{sec:ch31-Opportunities}

Robot software developers have often experienced a sense of
frustration when developing an application that is (nearly) the
same as several other releases for different projects, but they
have not been able to capture and exploit the commonalties. These
problems frequently and typically occur in four circumstances.

\begin{enumerate}
\item When the application migrates to a different robot platform.
Despite the differences in the mechanical structure, many
commonalties can be identified in the control architecture, such
as how to acquire sensory data or transmit commands to the
actuators.

\item When the application scales up from a single robot to a
multi-robot system or when it is restructured from centralized to
decentralized and distributed control paradigm. Despite the
differences in the communication mechanisms (shared memory or
networked environment) the interactions among the control modules
of the robot control architecture remain unchanged or show many
commonalties.

\item When the same robotic system is used in different environments or
for different tasks. Despite the differences in the control logic,
most of the robot capabilities and functionalities are common in
every application.

\item When the robotic system is build from the composition of
heterogeneous or legacy subsystems. Despite the differences in the
programming languages, concurrency models, or data interchange
formats, commonalties can be found in the semantics of the
components' behavior.
\end{enumerate}

Software frameworks are a possible solution to the above problems,
as illustrated in the following five sections.


\subsection{Frameworks for device access and control}
One vital task for every robot system is to interface with
hardware sensors and actuators. Familiar mass-market I/O devices
such as mice and joysticks were once specialist experimental
devices, comparable to the robot devices we build and buy today.
Each device had a unique interface, and software had to be
specially written to take advantage of it. Over time, as these
devices became common, their external interfaces have converged to
well-known logical standards. For user-level applications, the
data generated by a mouse is received as events from an abstract,
generic, "pointer" device. The hardware details are hidden behind
an interface that encapsulates the logical functionality of the
device. Other hardware, such as keyboards, disks and printers are
treated similarly. A significant fraction of the source code of a
operating system such as Windows or Linux is composed of such
"device driver" code.

Robotics devices are far more rare than mice and printers, but
recently we have seen a limited amount of commoditization in
research robots. The Pioneer 2 and 3 robots from Mobile Robots
Inc, and the SICK LMS laser scanner are good examples of devices
that are sold and used in the hundreds, all over the world. The
Pioneers ship with the software framework "ARIA" that contains
drivers for the Pioneer's hardware and the LMS. ARIA provides a
high-level, structured  (Object Oriented) environment for writing
robot controllers, and provides some powerful modules for doing
common tasks such as mapping and localization. It is a good
example of a system infrastructure framework: it provides a
complete system for robot programming and saves the controller
author from having to write device drivers herself.  The
"Mobility" APIs provide similar functionality for for RWI Robots,
and also support the SICK LMS. ARIA and Mobility are proprietary
systems that work only with their companies' robots. Indeed, they
provide a key part of the value of the robot to their customers:
they make programming robots much easier. But they target
different robots and their APIs are very different. If you move
your LMS from a Pioneer to an ATRV Jr. robot, you have to use
completely different code to access your range data.

Another approach is taken by the "Player" robot interface,
described in the Chapter by Richar Vaughan and Brian Gerkey. The
design philosophy of Player is to extend the computer's operating
system up to meet the robot controller. Like an operating system,
Player is designed to provide convenient access to hardware and
software resources, but otherwise keep out of the way. Robot
controllers using Player can be written in any programming
language, and devices can be accessed via a network socket or via
a linked C library. Thus Player places very few restrictions on
the user's robot control code.

The key feature of Player is that logically similar devices
present a similar interface. For example, most mobile robot bases
can accept velocity control commands  that are essentially [v,w]
tuples, and provide odometric position estimates that are
essentially [x,y,theta] pose tuples. Player supports both Pioneer
and RWI robots with exactly the same interface, along with several
other robots such as the old NOMAD range and the new Segway RMP.
Controllers that target Player's "position2d" abstract device
interface can control any of these robots without even being
recompiled. Similarly, Player's drivers for the SICK LMS and the
smaller, cheaper Hokuyo URG both implement the abstract "laser"
interface.

Robot controllers are written to target these abstract interfaces,
which means we can substitute any device with another that
provides the same logical functionality. So we can run the same
controller on different robot hardware or, more commonly, replace
the real robot with a simulator. Today, code is routinely
developed and tested using Player's simulation backends "Stage"
and "Gazebo" before being deployed on real robots. Often some
tweaks are needed when moving from simulation to reality, but the
convenience of prototyping in simulation overwhelms this extra
cost.

Player is a popular system, but its lack of constraints means that
it imposes no structure or high-level abstractions that can
usefully guide and facilitate controller design, such as those
provided by Mobility. Several projects have built on top of
Player, wrapping the device drivers and network transparency with
a more structured development environment, providing powerful
object hierarchies or visual programming interfaces. One example
is the MARIE system, described in the Chapter by Carle Cote et al.
This is an excellent of example of code re-use working in
practice, and is a very promising method for increasing the
efficiency of development in the community. Player and MARIE are
Free Software, so such code sharing is not hindered by proprietary
software issues. This is a key advantage in a small community with
no large software market.

\subsection{High-level communication Frameworks}

A very challenging robotic field that demands effective
integration techniques is that of personal robot companions.
Looking at the more abstract cognitive abilities of robot
companions, the integration of multiple modalities for achieving
human-like interaction capabilities requires flexible
communication frameworks that can be easily used by
interdisciplinary researchers. The interdisciplinary researchers
involved in integrated projects are neither integration nor
middleware experts, so ease of use is crucial for the successful
application of such a framework. Another aspect of research
prototypes is that specification changes occur frequently during
the development of these systems. Thus, the impact of interface
changes on the system architecture should be minimal to avoid
versioning problems. This, in turn, allows rapid prototyping and
iterative development that must be supported by a framework not
only for single modules but also for the integrated system.

A framework that addresses these requirements is the XCF-SDK
described in the Chapter "An Integration Framework for Developing
Interactive Robots" by J. Fritsch and S. Wrede. Instead of
targeting the whole range of issues involved in integrating
standard robotic software components, XCF provides a solution for
flexible data-driven component integration as well as event-driven
coordination by focusing on concepts from different middleware
architectures that are relevant for the domain of intelligent
systems research. This yields a lightweight API and provides
features from both black-box and white-box frameworks. Within XCF,
components are mainly described by the XML data they deliver to
other system modules either directly or mediated through instances
of an active XML database. Realizing an interceptor pattern and
utilizing XML features, rich support for system introspection is
provided in order to facilitate the debugging process.

Robotics4.NET is a message-oriented communication framework that
supports the implementation of control software for robots
designed to operate in human-based environments. It will be
presented in the Chapter by Antonio Cisternino et al. Its
peculiarity relies in its structure centered around the notion of
body of a robot. The proposed model has been inspired to the
biological structure of human body; this choice has been driven by
recent studies on the role of the body in the cognitive process of
human beings in the research area of Embodied AI. In this respect
the framework provides an XML-based, message-oriented
communication infrastructure that acts like the spine, connecting
organs to the brain. Body organs, called roblets provide an
abstract view of the underlying hardware subsystem they are
responsible for.

\subsection{Frameworks for functionality packaging}
Supporting specific target applications, such as common tasks in
robotics scenarios, is mandatory to allow robotic experts to focus
on the problem domain, instead of dealing with the entire problem
set of the application domain. It is also a key feature to
fostering reuse and generalization in application code. The
difficulty is, not to constrain the programmer in the choice of
its actual solution, or to impose restrictions on the kind of
solution for other problem areas of the application.

An application framework supports the development of applications
for common robotic tasks (such as video image processing or
behavior based reactive control), which manage the control flow
and organize the data flow for the task at hand, shielding the
application programmer from much of the intrinsic difficulties of
software development within a concurrent, distributed software
environment. Providing a framework for the tasks control and data
flow also facilitates to add centralized support for addressing
timeliness and scalability problems within the application.
Instead of imposing a fixed implementation, it provides callback
(in form of virtual methods) for integrating the application
programmers extensions for the solution of the problem in the
target scenario.

Unfortunately, the development of frameworks requires considerable
effort. But black-box style application framework components
require the interaction with the interfaces of a concrete robot
platform, which tends to limit their reusability. So the
integration of application frameworks with multi-platform robotics
middleware is desirable, to compensate the development effort by a
broad applicability of the application framework. It also allows
to reuse and utilize the provided infrastructure for the
implementation of the framework components.

The chapter by Utz et al. presents a video image processing
framework (VIP) that targets the problem of applying computer
vision in the field of real-time constraint autonomous mobile
robotics. It aims to provide an environment that allows the
developer to concentrate on the robot vision task without risking
to jeopardize the reactivity and performance of the system as a
whole.

The VIP framework features the above discussed different aspects
of robotics framework design. The white-box framework manages the
control flow and organizes the data flow, shielding the developer
from the subtle locking issues and memory management problems. In
order to meet the requirements of high-performance
real-time-constraint  robotics applications, it provides a
powerful sensor-triggered, on-demand processing model for image
operations. The framework also offers a set of tools that assist
in the debugging, assessment and tuning of the vision application.
Furthermore, the middleware integration provides a sophisticated
communication infrastructure for distributed robotics applications
as well as powerful facilities for the configuration and
parameterization that are exploited by the framework components to
enhance and their flexibility and reusability. A fast growing
library of black-box components for the framework confirms the
applicability of this application framework in very different
robotics scenarios.


\subsection{Frameworks for component integration}
Many existing software components can be found in robotics and it
would be beneficial to reuse them in an integrated fashion using
their initial implementation, saving time and avoiding introducing
errors when reimplementing everything from scratch. Unfortunately,
integration of existing software components is difficult knowing
that they are typically developed independently, following their
own set of requirements (e.g. timing, communication protocol,
programming language, operating system, objectives, applications).
Moreover, those components are often available "as is", with not
much support from their developers and with minimal documentation.
Reusability in this context is challenging but crucial for the
evolution of the field, avoiding becoming experts in all the
related areas that must be integrated.

System integration frameworks main objective is to support one or
many integration approaches (e.g. communication protocols, central
repository, remote procedure call, dynamic and static linkage) to
interconnect heterogeneous applications in a larger system with
its own set of concepts and requirements. Two key issues are
generally considered: system communications and heterogeneous
applications adaptation. To cope with those issues, system
integration frameworks facilitates the interconnection of
heterogeneous software components by providing flexible
implementations of communication protocols and mechanisms, and by
offering black-box and white-box frameworks to create
interoperable components that export the functionalities of
existing heterogeneous applications.

The Chapter by C�t� et al. presents MARIE, a system integration
framework that addresses three important issues in robotic
software development:

\begin{enumerate}
\item Being able to interconnect heterogeneous software components, from
legacy to novel "state of the art" software components.

\item Being able to support a wide range of communication protocols and
mechanisms to cope with the fact that there is no unified protocol
available, and no consensus has yet emerged from the robotic
software community.

\item Being able to support multiple sets of concepts and abstractions
within the integration frameworks, to help multidisciplinary team
members to collaborate without each member having to become an
expert in every aspect of robotics software development.
\end{enumerate}

MARIE's software architecture proposes a layered architecture,
composed of different white-box and black-box frameworks,
providing integration tools to help wrapping existing application
functionalities in reusable software blocks. Each layer abstracts
and manages different aspects needed to integrate heterogeneous
software components together, such as communications, data
handling, shared data types, distributed computing, low-level
operating system functions, component architecture, reusable
software blocks and system deployment, etc. With this approach,
MARIE offers a flexible and extendable software architecture
suitable in various integration situations, and allows software
developers to work at the required abstraction level to craft
reusable software blocks or to build robotic systems.


\section{Conclusions}
\label{sec:ch31-conclusions}



%%%%%%%%%%%%%%%%%%%%%%%%%%%%%%%%%%%%%%%%%%%%%%%%%%%%%%%%%%%%%%%%%%%%%%

%\printindex
%\end{document}
