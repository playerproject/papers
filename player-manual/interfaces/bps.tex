\section{{\tt bps} {\bf (Unfinished)}}

\subsection*{Overview}

The {\tt bps} device ({\bf b}eacon-based {\bf p}ositioning {\bf
s}ystem) determines the current global pose (position and
orientation) of the robot.  This pose is determined by combining
odometric measurements from the {\tt position} device with beacon
measurements from the {\tt laserbeacon} device.  Beacons must be
pre-placed at known locations, and these locations must be provided to
the {\tt bps} device via a configuration request.

Note that the {\tt bps} device is still experimental; while the external
interface is now more-or-less fixed, the internal structure is likely
to evolve.

\subsection*{Data}

The {\tt bps} device returns an estimate of the current position and
orientation of the robot.  The format for the data packet is given in
Table~\ref{table:bps_data}.

\begin{table}[ht]
\begin{center}
\begin{tabularx}{\columnwidth}{|l|l|X|}
\hline
Field & Type & Meaning\\
\hline
{\tt px} & {\bf int} & Current X position (mm)\\
{\tt py} & {\bf int} & Current Y position (mm)\\
{\tt pa} & {\bf int} & Current orientation (degrees)\\
{\tt ux} & {\bf int} & Current X uncertainty (mm)\\
{\tt uy} & {\bf int} & Current Y uncertainty (mm)\\
{\tt ua} & {\bf int} & Current orientation uncertainty (degrees)\\
{\tt reserved} & & Reserved\\
\hline
\end{tabularx}
\end{center}
\caption{{\em Format of data from the {\tt bps} device.}}
\label{table:bps_data}
\end{table}

\subsection*{Command}
The {\tt bps} device accepts no commands.

\subsection*{Configuration}

The {\tt bps} device accepts two configuration requests: 
{\tt setbeacon} and {\tt setlaser}.
The {\tt setbeacon} request adds a beacon to the robot's internal map;
the identity, position and orientation of the beacon must be supplied.
Note that beacons must be {\em unique}: i.e. no two beacons in the
environment can have the same identity.
The {\tt setlaser} request sets the pose of the laser in the
robot-centered coordinate system.  If the laser is placed near the
center of the robot, this offset can usually be ignored.
The format of the configuration request packets is shown in
Table~\ref{table:bps_config}.

\begin{table}[ht]
\begin{center}
\begin{tabularx}{\columnwidth}{|l|l|l|X|}
\hline
Request & Field & Type & Meaning \\
\hline
{\tt setbeacon} & {\tt subtype} & {\bf unsigned byte} & Request sub-type; must be 2. \\
                & {\tt id} & {\bf unsigned byte} & Beacon id. \\
                & {\tt px} & {\bf int} & Beacon X position (mm). \\
                & {\tt py} & {\bf int} & Beacon Y position (mm). \\
                & {\tt pa} & {\bf int} & Beacon orientation (degrees). \\
                & {\tt ux} & {\bf int} & Beacon X uncertainty (mm). \\
                & {\tt uy} & {\bf int} & Beacon Y uncertainty (mm). \\
                & {\tt ua} & {\bf int} & Beacon orientation uncertainty (degrees). \\
\hline
{\tt setlaser} & {\tt subtype} & {\bf unsigned byte} & Request sub-type; must be 3. \\
               & {\tt px} & {\bf int} & Laser X position (mm). \\
               & {\tt py} & {\bf int} & Laser Y position (mm). \\
               & {\tt pa} & {\bf int} & Laser orientation (degrees). \\
\hline
\end{tabularx}
\end{center}
\caption{{\em Format of the configuration requests from the {\tt bps} device.}}
\label{table:bps_config}
\end{table}
