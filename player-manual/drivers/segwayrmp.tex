\subsection*{Synopsis}
The {\tt segwayrmp} driver provides control of a Segway RMP (Robotic
Mobility Platform), which is an experimental robotic version of the Segway
HT (Human Transport), a kind of two-wheeled, self-balancing electric
scooter.

\subsection*{Interfaces}

\noindent Supported interfaces:

\begin{itemize}
\item {\tt position}
\item {\tt position3d}
\end{itemize}

\noindent Required devices:
\begin{itemize}
\item None
\end{itemize}

\noindent Supported configuration requests:
\begin{itemize}
\item For the {\tt position} interface:
\begin{itemize}
\item {\tt PLAYER\_POSITION\_MOTOR\_POWER\_REQ}
\item {\tt PLAYER\_POSITION\_GET\_GEOM\_REQ}
\item {\tt PLAYER\_POSITION\_RESET\_ODOM\_REQ}
\item {\tt PLAYER\_POSITION\_RMP\_VELOCITY\_SCALE}
\item {\tt PLAYER\_POSITION\_RMP\_ACCEL\_SCALE}
\item {\tt PLAYER\_POSITION\_RMP\_TURN\_SCALE}
\item {\tt PLAYER\_POSITION\_RMP\_GAIN\_SCHEDULE}
\item {\tt PLAYER\_POSITION\_RMP\_CURRENT\_LIMIT}
\item {\tt PLAYER\_POSITION\_RMP\_RST\_INTEGRATORS}
\item {\tt PLAYER\_POSITION\_RMP\_SHUTDOWN}
\end{itemize}
\item For the {\tt position3d} interface:
\begin{itemize}
\item {\tt PLAYER\_POSITION\_MOTOR\_POWER\_REQ}
\end{itemize}
\end{itemize}

\subsection*{Configuration file options}

\begin{center}
{\small \begin{tabular}{|l|l|c|l|}
\hline
Name & Type & Default & Meaning\\
\hline
{\tt canio} & string & {\tt "kvaser"} & The kind of underlying CAN I/O to
be used.\\

{\tt max\_xspeed} & int & {\tt 500} & Maximum allowed translational velocity
(mm/sec).\\

{\tt max\_yawspeed} & int & {\tt 40} & Maximum allowed angular velocity
(deg/sec).\\
\hline
\end{tabular}}
\end{center}

\subsection*{Notes}
\begin{itemize}
\item Because of its power, weight, height, and dynamics, the Segway RMP is
a potentially dangerous machine.  Be {\bf very} careful with it.
\item This driver is {\bf experimental}, as has {\bf not} been widely used
or extensively tested.  Use at your own risk.
\item Although the RMP does not actually support motor power control from 
software, for safety you must explicitly enable the motors using a
{\tt PLAYER\_POSITION\_MOTOR\_POWER\_REQ} request (this request is supported in
both {\tt position} and {\tt position3d} modes).
\item For safety, this driver will stop the RMP (i.e., send zero velocities)
if no new command has been received from a client in the previous 400ms or so.
Thus, even if you want to continue moving at a constant velocity, you must
continuously send your desired velocities.
\item Most of the configuration requests have not been tested.
\item The {\tt position3d} interface is entirely new and its use with this
driver has not been tested.
\item Currently, the only supported type of CAN I/O is {\tt "kvaser"},
which uses Kvaser, Inc.'s CANLIB interface library.  This library provides
access to CAN cards made by Kvaser, such as the LAPcan II.  However, the CAN 
I/O subsystem within this driver is modular, so that it should be pretty
straightforward to add support for other CAN cards.
\end{itemize}
