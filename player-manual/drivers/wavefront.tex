\subsection*{Synopsis}
The {\tt wavefront} driver implements a simple path planner for a planar
mobile robot.  The underlying planner, which uses wavefront propagation,
was written by Andrew Howard; the integration into Player was done by Brian
Gerkey.

This driver works in the following way: upon receiving a new position
target, a path is planned from the robot's current pose, as reported by
the underlying localize device.  The waypoints in this path are handed
down, in sequence, to the underlying position device. By tying everything together in this way, 
this driver offers the mythical ``global goto'' for your robot.

The planner first creates a configuration space of grid cells from the map that is given, 
treating black pixels as occupied and white pixels as free. The planner assigns a cost to each of the free cells based on
their distance to the nearest obstacle. The nearer the obstacle, the higher the cost. Beyond the 
max\_radius given by the user, the cost in the c-space cells is zero.

When the planner is given a new goal, it finds a path by working its way outwards from the goal cell, assigning plan costs to the cells as it expands (like a wavefront expanding outwards in water). The plan cost in each cell is dependant on its distance from the goal, as well as the obstacle cost assigned in the configuration space step. Once the plan costs for all the cells have been evaluated, the robot can simply follow the gradient of each lowest adjacent cell all the way to the goal.

In order to function effectively with an underlying obstacle avoidance algorithm, such as Vector Field Histogram, the planner only hands off waypoints, not the entire path. The wavefront planner finds the longest straight-line distances that don't cross obstacles between cells that are on the path. The endpoints of these straight lines become sequential goal locations for the underlying device driving the robot. 

\subsection*{Interfaces}

\noindent Supported interfaces:

\begin{itemize}
\item {\tt planner}
\end{itemize}

\noindent Required devices:
\begin{itemize}
\item {\tt position}
\item {\tt localize}
\end{itemize}

\noindent Supported configuration requests:
\begin{itemize}
\item None
\end{itemize}

\subsection*{Configuration file options}

\begin{center}
{\small \begin{tabularx}{\columnwidth}{|l|l|c|X|}
\hline
Name & Type & Default & Meaning\\
\hline
{\tt position\_index} & integer & {\tt 0} & Index of the underlying position device.\\
{\tt localize\_index} & integer & {\tt 0} & Index of the underlying
localize device.\\
{\tt map\_index} & string & none & Index of the map device containing map in which
to plan.\\
{\tt map\_scale} & float & none & Meters per pixels in map.\\
{\tt robot\_radius} & length & {\tt 0.15} & Radius of robot.\\
{\tt safety\_dist} & length & {\tt robot\_radius} & Distance to keep
between robot and obstacles.\\
{\tt max\_radius} & length & {\tt 1.0} & The maximum distance from obstacles for which the planner adds additional costs for traversing\\
{\tt dist\_penalty} & float & {\tt 1.0} & Fudge factor to discourage
cutting corners.\\
{\tt dist\_epsilon} & length & {\tt 3*robot\_radius} & Distance from target
position that will be considered acceptable.\\
{\tt angle\_epsilon} & angle & {\tt 10 degrees} & Angular difference from target
angle that will be considered acceptable.\\
\hline
\end{tabularx}}
\end{center}

\subsection*{Notes}
\begin{itemize}
\item This driver is new, not widely tested, and non-trivial to configure
and use.
\item There is currently no way to get feedback from the planner, such as:
the current list of waypoints, the lack of a feasible path, or the
achievement of the final goal.
\item The underlying position device must be capable of doing position
control (i.e., not velocity control), preferably with local obstacle
avoidance (a very good candidate is the {\tt vfh} driver).
\item The underlying localize device must have already converged to a
reasonable estimate of the robot's pose before targets are sent to this
driver (otherwise results will be unpredictable at best).
\item The target thresholds ({\tt dist\_epsilon} and {\tt angle\_epsilon})
should be {\em greater} than those thresholds in the underlying position
device, assuming it's {\tt vfh}.  Otherwise, the underlying driver thinks
the robot has reached a target, while the {\tt wavefront} driver is still
waiting.
\item You must compile Player having run configure with \--enable-wavefront option in order for the planner to work
\item There are three different coordinate frames of reference you will need to deal with when using the wavefront path planner:
\begin{itemize}
\item The Stage frame of reference, which has its (0,0) origin at the (0,0) pixel of the world map, or the bottom left hand corner. Your stage world file will specify a robot's start position in these coordinates.
\item The localize frame of reference, which has (0,0) at the center of your world map. If you are giving the robot a guess as to where the robot's starting position is, you will use this frame of reference. The wavefront planner and the amcl driver use this coordinate system.
\item The robot's frame of reference, or the frame of reference for a position driver, which will have its origin (0,0) at the point where the robot starts. This means that an interface such as VFH receives goal commands with reference to that point.
\end{itemize}
\end{itemize}


\subsection*{Example: Using the  {\tt wavefront } driver with Stage and the Playernav client}

The {\tt wavefront} driver can be used together with the  amcl driver
and the vfh obstacle avoidance driver to create autonomous navigation.  The
following example illustrates its use in the Stage simulation environment with the
hospital\_section.pnm map provided with Stage. In order to visualize the path planning, this example will
use Playernav, a client written to provide a GUI to command and monitor robots, display a map and the robot localization pose. 

\begin{itemize}
\item Start up the stage simulator from the directory {\em stagedirectory}/worlds:
\begin{verbatim}
$stage hospital_section.world
\end{verbatim}
The hospital\_section.world file should contain the following:

\begin{verbatim}
bitmap
( 
  file "hospital_section.pnm" #image is 1094 x 443 pixels
  resolution .05  #this is in meters/pixel
)
gui
(
  size [1000 600] #size of gui window
  origin [30 15] #center of the screen
  scale .07 #display scale
  grid [1 10] #minor and major grid lines
  showgrid 1 
)
define simple_robot position
(
  size [ 0.4 0.4 ]
  color "red"
  shape "circle" 
  laser ()
)
simple_robot
( 
  pose [ 10 15 0 ] 
  port 6665 
)
\end{verbatim}

If this was done correctly, the stage map should come up with a robot in place

\item Start up the player server from the directory {\em player directory}/config using the following command:
\begin{verbatim}
$player -p 7000 wavefront_hospitalsection.cfg
\end{verbatim}

Where the wavefront\_hosptialsection.cfg file contains:

\begin{verbatim}
position:0 ( driver "passthrough" port 6665)
laser:0 ( driver "passthrough" port 6665)
map:0 
(
  driver "mapfile" 
  filename "stage directory/worlds/hospital_section.pnm" 
  resolution 0.05 
  negate "1"
)
localize:0
(
  driver "amcl"
  odom_index 0
  laser_index 0
  laser_map_index 0
  init_pose [-17.35 3.92 0] #This is the robot's starting position
  init_pose_var [.2 .2 10]
  alwayson 0
 )
position:2
(
  driver "vfh"
  position_index 0
  laser_index 0
  cell_size 0.1
  window_diameter 61
  sector_angle 5
  safety_dist .15
  max_speed .5
  max_turnrate 75
  free_space_cutoff  1000000.0
  weight_desired_dir 5.0
  weight_current_dir 3.0
  distance_epsilon 0.3
  angle_epsilon 10
)
planner:0
(
  driver "wavefront"
  position_index 2
  localize_index 0
  map_index 0
  safety_dist 0.15
  max_radius 1
  dist_penalty 2.0
  distance_epsilon 0.5
  angle_epsilon 20
  alwayson 0
)
\end{verbatim}

\item Now you can start the playernav client to command the robot and see the planned paths. From the directory {\em playerdirectory}/utils/playernav run:

\begin{verbatim}
$./playernav localhost:7000
\end{verbatim}

You will now get a new window with the map and the robot's position. You can drag and drop the robot with the left mouse button to 
give it a guess as to its position, but with the above configuration file, it will already be in the correct spot. Using the right mouse button
you can drag and drop the goal triangle off the robot, to the desired goal and orientation.

\end{itemize}




