\subsection*{Synopsis}
The {\tt udpbroadcast} driver provides a mechanism whereby Player clients can
communicate with other Player clients, even when those clients are connected
to different Player servers.  Any message sent to a {\tt udpbroadcast} device
will be received by all {\tt udpbroadcast} devices that are on the same
physical network (including the sending device).  The underlying transport
mechanism is based on broadcast UDP sockets (see Notes below).

\subsection*{Interfaces}

\noindent Supported interfaces:

\begin{itemize}
\item {\tt comms}
\end{itemize}

\noindent Required devices:
\begin{itemize}
\item None.
\end{itemize}

\noindent Supported configuration requests:
\begin{itemize}
\item None.
\end{itemize}


\subsection*{Configuration file options}

\begin{center}
{\small \begin{tabular}{|l|l|c|l|}
\hline
Name & Type & Default & Meaning\\
\hline
{\tt addr} & string & {\tt "10.255.255.255"} & The broadcast address to be
used.\\
{\tt port} & integer & {\tt 6013} & The broadcast port to be used.\\
\hline
\end{tabular}}
\end{center}

\subsection*{Notes}

\begin{itemize}
\item Make sure your network supports broadcasting, and that your
sys-admin wont cut you off for trying!  Broadcasting is best done
on private networks.
\item The default broadcast address is 10.255.255.255, port 6013
(i.e. it assumes you have a 10.*.*.* network with netmask 255.0.0.0).
The broadcast address and port are configurable in Player's configuration
file.
\item There is no ``simulated'' {\tt udpbroadcast} driver in Stage; the real
{\tt udpbroadcast} driver is always used.  In this case, the default 
broadcast address is 127.255.255.255 (the loopback device).  At present,
there is no way of changing the default value.
\item There is no guarantee that messages will be delivered, or 
that they will be delivered in the exact order they were transmitted.
\item The {\tt udpbroadcast} driver is meant for transmitting small packets
only: dont try to use it for passing full-color images around at 30Hz!
You will flood your network and overflow both incoming and outgoing queues
in the {\tt udpbroadcast} driver.
\end{itemize}

%and you dont have a 10.0.0.0 network, you can easily create a dummy one.  
%As root, type the following:
%  \begin{quote}
%  {\tt ifconfig dummy0 10.0.0.1 netmask 255.0.0.0 broadcast 10.255.255.255}
%  \end{quote}
%Your broadcast packets will now loop back to you.
