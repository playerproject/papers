
\subsection*{Synopsis}

The {\tt passthrough} driver acts as a {\em client} to another Player
server; it returns data generated by the remote server to client
programs, and send commands from the client programs to the remote
server.  In this way, one Player server can pretend to have devices
that are actually located at some other location in the network (i.e.,
owned by some other Player server).  Thus, the {\tt passthrough}
driver makes possible two important capabilities:
\begin{itemize}
\item Data from multiple robots can be aggregated in a single Player
server; client programs can then talk to more than one robot through a
single connection.
\item Computationally intensive drivers can be moved off the robot and
onto a workstation.  Client programs connect to the workstation rather
that the robot, but are otherwise unchanged.
\end{itemize}
See the below for some examples of the {\tt passthrough} driver in
action.

\subsection*{Interfaces}

The {\tt passthrough} driver will support any of Player's interfaces,
and can connect to any Player device.


\subsection*{Configuration file options}

\begin{center}
{\small \begin{tabularx}{\columnwidth}{|l|l|c|X|}
\hline
Name & Type & Default & Meaning\\
\hline
{\tt host} & string & {\tt localhost} & Host name for the machine running the remote
Player server.\\
{\tt port} & integer & 6665 & Port number for remote server.\\
{\tt index} & integer & 0 & Index of the device on the remote server.\\
\hline
\end{tabularx}}
\end{center}

%\subsection*{Notes}



\subsection*{Example: Controlling multiple robots through a single connection}

The {\tt passthrough} driver can be used to aggregate devices from multiple
robots into a single server.  The following example illustrates the general
method for doing 
%
\begin{itemize}
\item Imagine that we have to robots named {\tt bee} and {\tt bug}.
On each robot, start a Player server with the following configuration
file:
  \begin{verbatim}
  position:0 (driver "p2os_position")
  laser:0 (driver "sicklms200")
  \end{verbatim}
In this example, the robots are assumed to be Pioneer's with SICK
laser range-finders. 
\item Now imagine that we have a workstation named {\tt orac}.  On
this workstation, start another instance of Player with the following
configuration file:
  \begin{verbatim}
  position:0 (driver "passthrough" host "bee" port 6665 index 0)
  laser:0 (driver "passthrough" host "bee" port 6665 index 0) 
  position:1 (driver "passthrough" host "bug" port 6665 index 0)
  laser:1 (driver "passthrough" host "bug" port 6665 index 0) 
  \end{verbatim}
A client connecting to {\tt orac} will see four devices: two {\tt
position} devices and two {\tt laser} devices.  Both robots can now be
controlled through a single connection to {\tt orac}.
\end{itemize}


\subsection*{Example: Shifting computation}

Computationally expensive drivers (such as {\tt adaptive\_mcl}) can be
shifted off the robot and onto a workstation.  The basic method is
a straight-forward variant of the example given above.
%
\begin{itemize}
\item Imagine that we have a robot named {\tt bee}.  On {\tt bee}, run
the Player server with this configuration file:
  \begin{verbatim}
  position:0 (driver "p2os_position")
  laser:0 (driver "sicklms200")
  \end{verbatim}
The robot is assumed to be a Pioneer with a SICK laser range-finder. 
\item Now imagine that we have a workstation named {\tt orac}.  On
this workstation, start another instance of Player with the
following configuration file:
  \begin{verbatim}
  position:0 (driver "passthrough" host "bee" port 6665 index 0)
  laser:0 (driver "passthrough" host "bee" port 6665 index 0) 
  localize:0 (driver "adaptive_mcl" position_index 0 laser_index 0 ...)
  \end{verbatim}
(see Section \ref{sect:amcl_driver} for a detailed description of the
additional setings for the {\tt adaptive\_mcl} driver).  Clients
connecting to this server will see a robot with {\tt position}, {\tt
laser} and {\tt localize} devices, but all of the heavy computation
will be done on the workstation.
\end{itemize}



\subsection*{Example: Using the {\tt adaptive\_mcl} driver with Stage}

Some newer drivers, such as the {\tt adaptive\_mcl} driver, are not
supported natively in Stage.  For these drivers users must employ a
second Player server configured to use the {\tt passthrough} driver.
The basic procedure is as follows.
%
\begin{itemize}
\item Start Stage with a world file something like this:
  \begin{verbatim}
  ...
  position (port 6665 laser ())
  ...
  \end{verbatim}
Stage will create one robot (position device) with a laser, and
will start a Player server that listens on port 6665.
\item Start another Player server using the command
  \begin{verbatim}
  player -p 7000 amcl.cfg
  \end{verbatim}
where the configuration file {\tt amcl.cfg} looks like this (see
Section \ref{sect:amcl_driver} for a detailed description of the
setings for the {\tt adaptive\_mcl} driver):
  \begin{verbatim}
  position:0 (driver "passthrough" port 6665 index 0)
  laser:0 (driver "passthrough" port 6665 index 0) 
  localize:0 (driver "adaptive_mcl" position_index 0 laser_index 0 ...)
  \end{verbatim}
The second Player server will start up and listen on port 7000;
clients connecting to this server will see a robot with {\tt
position}, {\tt laser} and {\tt localize} devices.
\end{itemize}




