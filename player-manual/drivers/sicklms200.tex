\subsection*{Synopsis}
The {\tt sicklms200} driver is used to control a SICK LMS-200 laser
range-finder.

\subsection*{Interfaces}

\noindent Supported interfaces:
\begin{itemize}
\item {\tt laser}
\end{itemize}

\noindent Required devices:
\begin{itemize}
\item None.
\end{itemize}

\noindent Supported configuration requests:
\begin{itemize}
\item \verb+PLAYER_LASER_GET_GEOM+
\item \verb+PLAYER_LASER_GET_CONFIG+
\item \verb+PLAYER_LASER_SET_CONFIG+
\end{itemize}


\subsection*{Configuration file options}

\begin{center}
{\small \begin{tabularx}{\columnwidth}{|l|l|c|X|}
\hline
Name & Type & Default & Meaning\\
\hline
{\tt pose} & tuple & {\tt [0.0 0.0 0.0]} & The mounted pose of the laser (in mm,
mm, degrees)\\
{\tt delay} & integer & 0 & Startup delay on the laser, in seconds (set this to 35 if you
have a newer generation Pioneer whose laser is switched on when the serial port is open). \\
{\tt port} & string & {\tt "/dev/ttyS1"} & The serial port to be used\\
{\tt rate} & integer & {\tt 38400} & Baud rate to use when talking to
laser; should be one of 9600, 38400, 500000.\\
{\tt resolution} & integer & {\tt 50} & The angular scan resolution to be used
(in units of 0.01 degrees)\\
{\tt range\_res} & integer & {\tt 1} & The range resolution mode of the laser.  Set to 1 to get
1~mm resolution with 8~m max range; set to 10 to get 1~cm resolution with 80~m range; set to
100 to get 10~cm resolution with 800~m max range.\\
{\tt invert} & integer & {\tt 0} & Set this flag if the laser is mounted upside-down; the range
and bearing results will be flipped so the laser appears to be right-way-up.\\
\hline
\end{tabularx}}
\end{center}

\subsection*{Notes}

\begin{itemize}
\item This driver likely only works in Linux.
\end{itemize}
