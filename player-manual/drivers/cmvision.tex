
\subsection*{Synopsis}
CMVision (Color Machine Vision) is a fast color-segmentation (aka
blob-finding) software library.  CMVision was written by Jim Bruce at CMU
and is Freely available under the GNU GPL:\\
\indent {\tt http://www-2.cs.cmu.edu/$\sim$jbruce/cmvision/}\\
But you don't have to download CMVision yourself, because Player's 
{\tt cmvision} driver includes the CMVision code.  The {\tt cmvision}
driver provides a stream of camera images to the CMVision code and
assembles the resulting blob information into Player's {\tt blobfinder}
data format.

The frame-grabbing portion of the {\tt cmvision} driver is modular,
allowing the user to select the source of camera images.  Currently, the
following sources are supported (see below for how to select the capture
source).  Note that support for each source is compiled only if the
required libraries and/or kernel features are detected.
\begin{itemize}
\item IEEE 1394 (aka Firewire) cameras; requires the {\tt libraw1394} and
{\tt libdc1394} development packages (both are Freely available)
\item Video4Linux (aka V4L) cameras; requires V4L headers
\item Video4Linux2 (aka V4L2) cameras; requires V4L2 support in your
kernel ({\em currently disabled})
\item A video source that supports Player's internal {\tt camera}
interface, such as the Gazebo camera driver
\end{itemize}

\subsection*{Interfaces}

\noindent Supported interfaces:
\begin{itemize}
\item {\tt blobfinder}
\end{itemize}

\noindent Required devices:
\begin{itemize}
\item None.
\end{itemize}

\noindent Supported configuration requests:
\begin{itemize}
\item None.
\end{itemize}

\subsection*{Configuration file options}
\begin{center}
{\small \begin{tabularx}{\columnwidth}{|l|l|c|X|}
\hline
Name & Type & Default & Meaning\\
\hline
{\tt capture} & string & {\tt "1394"} & Capture source (should be {\tt
"1394"}, {\tt "V4L2"}, {\tt "V4L"}, or {\tt "camera"})\\
{\tt colorfile} & string & {\tt ""} & (absolute?) path to the CMVision configuration file.\\
{\tt height} & integer & {\tt 240} & Height of the camera images (pixels).\\
{\tt width} & integer & {\tt 320} & Width of the camera images (pixels).\\
\hline
\end{tabularx}}
\end{center}

\subsection*{Notes}
\begin{itemize}
\item This driver (or at least its underlying video capture code) only works in
Linux.
\item Consult the CMVision documentation for details on writing a CMVision
configuration file.
\end{itemize}

