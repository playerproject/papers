\subsection*{Synopsis}
The {\tt er1} driver provides position control of the Evolution Robotics' ER1 and ERSDK robots.

\subsection*{Interfaces}

\noindent Supported interfaces:
\begin{itemize}
\item {\tt position}
\end{itemize}

\noindent Required devices:
\begin{itemize}
\item None.
\end{itemize}

\noindent Supported configuration requests:
\begin{itemize}
\item {\tt PLAYER\_POSITION\_GET\_GEOM\_REQ}
\item {\tt PLAYER\_POSITION\_MOTOR\_POWER\_REQ}
\end{itemize}

\subsection*{Configuration file options}

\begin{center}
{\small \begin{tabularx}{\columnwidth}{|l|l|c|X|}
\hline
Name & Type & Default & Meaning\\
\hline
{\tt port} & string & {\tt "/dev/usb/ttyUSB1"} & The serial port to be used\\
\hline
{\tt axle} & float & {\tt 0.38} & The distance between the motorized wheels\\
\hline
{\tt motor\_dir} & -1,1 & {\tt 1} & Direction of the motors, if the left motor is plugged in to the motor 1 port on the RCM, put -1 here instead\\
\hline
{\tt debug} & 0,1 & {\tt 0} & Put a 1 here if you want to see debug messages\\
\hline
\end{tabularx}}
\end{center}


\subsection*{Notes}
\begin{itemize}
\item This driver is new and not thoroughly tested.  The odometry cannot be trusted to give accurate readings.

\item You will need a kernel driver to allow the serial port to be seen.  This driver, and news about the player driver can be found at "http://www-robotics.usc.edu/~dfseifer/project-erplayer.php".

\item TODO: split this driver similar to the way that p2os is split, one main body device, and sub-devices like position, and IR which inherit from this main class.  Implement power interface.

\item NOT DOING: I don't have a gripper, if someone has code for a gripper, by all means contribute it.  It would be welcome to the mix.
\end{itemize}
