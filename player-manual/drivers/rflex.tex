\subsection*{Synopsis}
The {\tt rflex\_*} family of drivers are used to control RWI robots by directly
communicating with RFLEX onboard the robot (i.e., Mobility is bypassed).
To date, these drivers have been tested on an ATRV-Jr, but they {\em should}
work with other RFLEX-controlled robots: you will have to determine some
parameters to set in the config file, however.

As of March 2003 these drivers have been modified to support the b21r robot,
Currently additional support has been added for the power interface and bumper interface. For the pan tilt unit on the b21r please refer to the ptu46 driver. -- Toby Collett

\subsection*{Interfaces / Configuration requests}
Although all the RFLEX interaction is actually done in a single thread, the
different devices are accessed through different Player drivers:
\begin{itemize}
\item {\tt rflex\_position}:
  \begin{itemize}
  \item Interface: {\tt position} (see Section~\ref{sect:position})
  \item Configurations: {\tt VELOCITY\_MODE}, {\tt SET\_ODOM},
                        {\tt GET\_GEOM}, {\tt MOTOR\_POWER},
                        {\tt RESET\_ODOM}
  \end{itemize}
\item {\tt rflex\_sonar}:
  \begin{itemize}
  \item Interface: {\tt sonar} (see Section~\ref{sect:sonar})
  \item Configurations: {\tt GET\_GEOM}, {\tt POWER}
  \end{itemize}
\item {\tt rflex\_power}:
  \begin{itemize}
  \item Interface: {\tt power} (see Section~\ref{sect:power})
  \item Configurations: none
  \item Notes: The power driver seems to report a different value than that on the LCD on the robot so
  an offset can be added in the configuration file.
  \end{itemize}
\item {\tt rflex\_bumper}:
  \begin{itemize}
  \item Interface: {\tt bumper} (see Section~\ref{sect:bumper})
  \item Configurations: {\tt GET\_GEOM}
  \end{itemize}
\item {\tt rflex\_ir}:
  \begin{itemize}
  \item Interface: {\tt ir} (see Section~\ref{sect:ir})
  \item Configurations: {\tt POSE}, {\tt POWER}
  \end{itemize}

\end{itemize}
Some generic devices (e.g. {\tt aio} and {\tt dio}) may be available, but are
untested.

\subsection*{Configuration file options}
For example configuration
files, see {\tt umass\_ATRVJr.cfg}, {\tt umass\_ATRVMini.cfg} and
{\tt b21r\_rflex\_lms200.cfg}.

\textbf{IMPORTANT: }
Due to a number of initilisation issues relating to the multipart nature of the rflex driver the
configuration option \emph{rflex\_done} must be set to 1 in the last rflex driver in the config file. This will
cause the server to wait until all the rflex driver options have been parsed before launching its main thread.
\emph{The driver will hang if you do not speify this value}

\subsection*{rflex\_position}
\begin{center}
{\small \begin{tabular}{|l|l|l|l|}
\hline
Name & Type & Default & Meaning\\
\hline
{\tt rflex\_serial\_port} & string & {\bf none} & Serial port connected to RFlex device. See note 5.\\
{\tt mm\_length} & string & {\bf none} & Length of the robot in millimeters\\
{\tt mm\_width} & string & {\bf none} & Width of the robot in millimeters\\
{\tt odo\_distance\_conversion} & string & {\bf none} & Odometry conversion. See Note 1.\\
{\tt odo\_angle\_conversion} & string & {\bf none} & Odometry conversion. See Note 2.\\
{\tt default\_trans\_acceleration} & string & {\bf none} & Set translational acceleration, in mm.\\
{\tt default\_rot\_acceleration} & string & {\bf none} & Set rotational acceleration, in radians.\\
\hline
\end{tabular}}
\end{center}

\subsection*{rflex\_sonar}
\begin{center}
{\small \begin{tabular}{|l|l|l|l|}
\hline
Name & Type & Default & Meaning\\
\hline
{\tt rflex\_serial\_port} & string & {\bf none} & Serial port connected to RFlex device. See note 5.\\
{\tt range\_distance\_conversion} & string & {\bf none} & Sonar range conversion factor. See Note 7.\\
{\tt sonar\_age} & string & {\bf none} & Prefiltering parameter. See Note 3.\\
{\tt max\_num\_sonars} & string & {\bf none} & See Note 4.\\
{\tt num\_sonars} & string & {\bf none} & See Note 4.\\
{\tt num\_sonar\_banks} & string & {\bf none} & See Note 4.\\
{\tt num\_sonars\_possible\_per\_bank} & string & {\bf none} & See Note 4.\\
{\tt num\_sonars\_in\_bank} & string & {\bf none} & See Note 4.\\
{\tt mmrad\_sonar\_poses} & string & {\bf none} & Sonar positions and directions.  See Note 6.\\
\hline
\end{tabular}}
\end{center}

\subsection*{rflex\_bumper}
\begin{center}
{\small \begin{tabular}{|l|l|l|l|}
\hline
Name & Type & Default & Meaning\\
\hline
{\tt rflex\_serial\_port} & string & {\bf none} & Serial port connected to RFlex device. See note 5.\\
{\tt bumper\_count} & int & {\bf none} & Number of bumper panels\\
{\tt bumper\_def} & float tuple & {\bf none} & Tuple of geometry for each bumper\\
{\tt rflex\_bumper\_address} & int & {\bf 64} & The base address of first bumper in the DIO address range\\
\hline
\end{tabular}}
\end{center}

\subsection*{rflex\_ir}
\begin{center}
{\small \begin{tabular}{|l|l|l|l|}
\hline
Name & Type & Default & Meaning\\
\hline
{\tt rflex\_serial\_port} & string & {\bf none} & Serial port connected to RFlex device. See note 5.\\
{\tt rflex\_base\_bank} & int & {\bf 0} & Base IR Bank\\
{\tt rflex\_bank\_count} & int & {\bf 0} & Number of banks in use\\
{\tt rflex\_banks} & int tuple & {\bf [0]} & Number of IR sensors in each bank\\
{\tt pose\_count} & int & {\bf 0} & Total Number of IR sensors\\
{\tt ir\_min\_range} & int & {\bf 0} & Min range of ir sensors (mm) (Any range below this is returned as 0)\\
{\tt ir\_max\_range} & int & {\bf 0} & Max range of ir sensors (mm) (Any range above this is returned as max)\\
{\tt rflex\_ir\_calib} & float tuple & {\bf [1 1]} & IR Calibration data (see Note 8)\\
{\tt poses} & float tuple & {\bf [0]} & x,y,theta of sensors (mm, mm, deg)\\
\hline
\end{tabular}}
\end{center}

\subsection*{rflex\_power}
\begin{center}
{\small \begin{tabular}{|l|l|l|l|}
\hline
Name & Type & Default & Meaning\\
\hline
{\tt rflex\_serial\_port} & string & {\bf none} & Serial port connected to RFlex device. See note 5.\\
{\tt rflex\_power\_offset} & int & {\bf 0} & The calibration constant for the power calculation in decivolts\\
\hline
\end{tabular}}
\end{center}


\subsection*{Notes}
\begin{enumerate}
\item Since the units used by the Rflex for odometry appear to be completely
arbitrary, this coefficient is needed to convert to millimeters:
mm = (rflex units) / (odo\_distance\_conversion).  These arbitrary units also
seem to be different on each robot model. I'm afraid you'll have to determine
your robot's conversion factor by driving a known distance and observing the
output of the RFlex.
\item Conversion coefficient for rotation odometry: see {\tt odo\_distance\_conversion}. Note that heading is re-calculated by the Player driver since the RFlex is not very accurate in this respect. See also Note 1.
\item Used for prefiltering: the standard Polaroid sensors never return values that are closer than the closest obstacle, thus we can buffer locally looking for the closest reading in the last "sonar\_age" readings. Since the servo tick here is quite small, you can still get pretty recent data in the client.
\item These values are all used for remapping the sonars from Rflex indexing
to player indexing. Individual sonars are enumerated 0-15 on each board, but
at least on my robots each only has between 5 and 8 sonar actually attached.
Thus we need to remap all of these indexes to get a contiguous array of
N sonars for Player.
  \begin{itemize}
    \item {\tt max\_num\_sonars} is the maximum enumeration value+1 of all sonar
    meaning if we have 4 sonar boards this number is 64.
    \item {\tt num\_sonars} is the number of physical sonar sensors - meaning the number of ranges that will be returned by Player.
    \item {\tt num\_sonar\_banks} is the number of sonar boards you have.
    \item {\tt num\_sonars\_possible\_per\_bank} is probobly 16 for all robots,
    but I included it here just in case. this is the number of sonar that can
    be attached to each sonar board, meaning the maximum enumeration value
    mapped to each board.
    \item {\tt num\_sonars\_in\_bank} is the nubmer of physical sonar attached
    to each board in order - you'll notice on each the sonar board a set of dip
    switches, these switches configure the enumeration of the boards
    (ours are 0-3)

  \end{itemize}

\item The first RFlex device (position, sonar or power) in the config file must include this option, and only the first device's value will be used.
\item This is about the ugliest way possible of telling Player where each sonar is mounted.  Include in the string groups of three values:{\tt "x1 y1 th1 x2 y2 th2 x3 y3 th3 ..."}.  x and y are mm and theta is radians, in Player's robot coordinate system.
\item Used to convert between arbitrary sonar units to millimeters: mm = sonar units / range\_distance\_conversion.
\item Calibration is in the form Range = $(Voltage/a)^b$ and stored in the tuple as [a1 b1 a2 b2 ...] etc for each ir sensor.
\end{enumerate}

